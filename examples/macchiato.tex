\documentclass[12pt]{article}

% Imports the catppuccino macchiato theme
% from the directory above. Actual implementation
% wouldn't need the import package unless the theme
% and the document are in different directories.
\usepackage{import}
\import{../}{catppuccin_macchiato.sty}

% Removes padding above title
\usepackage{titling}
\setlength{\droptitle}{-10em}

% Margin package
\usepackage[margin=1.5in]{geometry}

% Font package
\usepackage[T1]{fontenc}
\usepackage{inconsolata}

% Code snippets package
\usepackage{listings}
% Add code highlighting specific for my Ruby on Rails snippet
% You'd want to follow something similar below to add your own keyword highlighting
\lstdefinestyle{ruby_on_rails}{
  language={Ruby},
  breaklines=true,
  showstringspaces=false,
  breakatwhitespace=true,
  stringstyle = {\color{macchiatoGreen}},
  commentstyle={\color{macchiatoOverlay1}},
  basicstyle = {\small\color{macchiatoText}\ttfamily},
  keywordstyle = {\color{macchiatoMauve}},
  keywordstyle = [2]{\color{macchiatoBlue}},
  keywordstyle = [3]{\color{macchiatoYellow}},
  keywordstyle = [4]{\color{macchiatoLavender}},
  keywordstyle = [5]{\color{macchiatoPeach}},
  keywordstyle = [6]{\color{macchiatoTeal}},
  otherkeywords = {<, ||, =, ?},
  morekeywords = [2]{new, create, present, email, description, creator, protect_from_forgery, before_action},
  morekeywords = [3]{PageController, ApplicationController, Page},
  morekeywords = [4]{@page},
  morekeywords = [5]{exception, do_some_for_pages, @page, @admin},
  morekeywords = [6]{<, ||, =, ?},
}

% Set the title and author, utilizing macchiatoPink
\title{ \Huge \textbf{\textcolor{macchiatoPink}{Catppuccin Macchiato Theme for} \textcolor{macchiatoLavender}{\LaTeX{} Documents}} \vspace{-3em}}
\date{}

% Start our dock
\begin{document}

\maketitle

\section{\textcolor{macchiatoSky}{Simple Paragraph}}
\textcolor{macchiatoYellow}{This is what a simple paragraph looks like.} You can see that all the colors in this pallet complement each other and maintain an elegance all within the same proximity. The default text color is called \textbf{\textcolor{macchiatoGreen}{macchiatoText}} while the background color is called \textbf{\textcolor{macchiatoGreen}{macchiatoBase}}.

\section{\textcolor{macchiatoSky}{Math}}

You even can make math formulas, equations, and examples look nicer and more legible:


\[\large{
    \textcolor{macchiatoBlue}{\lim_{{n \to \infty}} \textcolor{macchiatoPeach}{\int_{\textcolor{macchiatoRed}{a}}^{\textcolor{macchiatoRed}{b}} \frac{\textcolor{macchiatoGreen}{\sin}(\textcolor{macchiatoYellow}{nx})}{\textcolor{macchiatoYellow}{x}} \, \textcolor{macchiatoTeal}{dx}}}
}\]

\tiny{(Or just color dump like I did)}

\section{\textcolor{macchiatoSky}{Code Snippets}}

\begin{lstlisting}[language=Ruby,style=ruby_on_rails, caption={A ruby on rails code sample}]
class PageController < ApplicationController
  protect_from_forgery with: :exception
  before_action :do_some_for_pages

  '''
  This action creates a new page
  '''
  def new
    @page = Page.new
    creator = current_user || @admin

    # Check if a creator is present
    if creator.present?
      @page.creator = creator.email
    end

    @page.description = 'This is the default description'
  end
end
\end{lstlisting}

\end{document}


